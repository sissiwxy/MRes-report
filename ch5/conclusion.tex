\chapter{Conclusion and outlook}

In this report, we have introduced a theoretical framework for describe light scattering in linear, complex nanophotonic networks. We have investigated the transport properties of such networks using simulation. Some key insights include the transition from diffusive to localised regime by increasing disorder (node number density) of a network, as well as the relation between mode size (IPR) and transmission. We set up a quantum optical model for the system, and studied simple examples of quantum interference. 

There are several avenues for near-term future work on the simulation side. First of all, we can conduct a more thorough study of localisation in complex networks, such as by looking at transmission level spacing \cite{Gaio2017}. It might also be worthwhile to look at phase, and not just intensity effects in the transmitted signal, although this is harder to access experimentally. We can also look at a wider range of network geometries beyond Delaunay and Voronoi graphs.

In the long term, a promising route is to incorporate the scattering matrix formalism, which will make our model more realistic and more robust. We can also investigate the effect of tuning (e.g. adding phase gates) in the network, achieved by photo-optic effects in materials--this is in line with our goal to use these networks for light state synthesis and engineering. Finally, it would be interesting to look at time-response of the network, by considering pulses which are a superpositions of Fourier $k$ components.

On the experimental front, the obvious next step is to fix the issue of reliable coupling into the networks and collect measurement results we can compare with theory. One possible way to do this is by switching from grating couplers to in-situ 3D printed couplers \cite{Dietrich2018}.

This project is an interdisciplinary effort involving nanophotonics, theoretical solid-state physics, and the mathematics of complex networks. It is still in its very early stages, but in time we hope it will provide a novel platform for controlling classical and quantum light.